\documentclass[]{article}
\usepackage{amsmath}
\usepackage{ amssymb }

%opening
\title{}
\author{Peter Harding}

\begin{document}

\maketitle

\begin{abstract}
Deakin Uni Physics for the Life Sciences Notes
\end{abstract}

\section{Constants}

\begin{align*}
	c &= 3.00 \times 10^9 \  m/sec \\
	e &= 1.80 \times 10^{-19} \  C \\
	g &= 9.8 \  m/sec^2 \\
	1 \  \text{atm} &= 1.01 \times 10^5 Pa = 760 \  mm Hg \\
	\text{Coulomb's} \   K &= 9 \times 10^9 \  Nm^2 C^{-2} \\
	\text{Speed of Sound} &= 343 \  m/sec \\
	1 \text{Cal} &= 4.186 \  J \\
	1 \text{eV} &= 1.60 \times 10^-19 \  J \\
	\text{Electron Mass} &= 9.11 \times 10^{-31} \  Kg \\
	\text{Proton Mass} &= 1.67 \times 10^{-17} \  Kg \\
	\text{Atomic Mass Unit} &=  1.67 \times 10^{-17} \  Kg \\
%% 	N_A &= 6.02 \times 10^23 \text{particles/mol) \\
	\epsilon_0 &= 8.85 \times 10^{-12} \  C^2/Nm^2 \\
	R &= 6.31 \  J/Mol-K \\ 
	\text{Threshold of hearing} &= l_0 = 1.0 \times 10^{-12} \  W/m^2 \\
    1 \  \text{curie} &= 3.7 \times 10^{10} \  Bq \\
    k_g &= 1.38 \times 10^{-23} \  J/K \\
    R &= 8.31 \  J/mol-K \\
    \text{Speed of Light} &= 3.0 \times 10^{8} \  m/sec \\
    \hbar &= 1.05 \times 10^{-34} \  J * s = 6.58 \times 10^{-18} \  eV*s \\
    \text{Density of Water} &= 1000 \  kg/m^3
\end{align*}

\section{Conversion Formulae}
\begin{align*}
	T &= T_c + 273 \\
	T(^\circ C) &= \frac{5}{9}[T(^\circ F) - 32 ^\circ] \\
	n &= \frac{M \  (\text{in grams})}{M_{mol}} = \frac{N}{N_A}
\end{align*}

\subsection{Trigonometry}

\begin{align*}
	cosine &= \frac{adjacent}{hypotenuse} \\
	sine &= \frac{opposite}{hypotenuse} \\
	tangent &= \frac{opposite}{adjacent}
\end{align*}

\subsection{Pythagorean Theorem}

\begin{align*}
	a^2 &= b^2 +c^2
\end{align*} 

\newpage


\section{Formulae}

\subsection{Area and Volume Formulae}

\begin{align*}
	\text{Circle} \qquad  A &= 2 \pi r  & V &= \pi r^2 \\
	\text{Cylinder} \qquad  A &= 2 \pi r h &  V &= \pi r^2 h \\
	\text{Sphere} \qquad  A &= 4 \pi r^2 &  V &= \frac{4}{3} \pi r^3 \\
	\text{Cube} \qquad  A &= 6 h^2 &  V &= h^2 \\
\end{align*} 

\subsection{Length of a vector}

\begin{align*}
	\text{Length of a Vector} \qquad \left|  V \right| &= \sqrt{V^2_x + V^2_y}
\end{align*}

\newpage



\section{Kinemetics}

\begin{align*}
	\text{Equations of Motion} \qquad  \Delta x &= x_j = x_j \\
	\nu &= \frac{\Delta{x}}{\Delta{t}} \\
	a &= \frac{\Delta v}{\Delta T} \\
	v^2_f &= v^2_i + 2 a \Delta X \\
	v_f &= v_i + a \Delta t \\
	\Delta x &= v_i \Delta t  + \frac{1}{2} a \Delta t^2 \\
\\
\\
	\text{Net Force} \qquad F_{net} &= F_1 + F_2 + ... + F_N \\
	\\
	\text{Newton's Second Law} \qquad F &= m a \\
	\\
	\text{Drag Force} \qquad D &= \frac{1}{2} C_d \rho v^2 A \\
	\\
	\text{Friction Force} \qquad f_{s,max} &= \mu_s n & f_k &= \mu_k n \\
	\\
	\text{Spring Force  (Hook's Law)} \qquad F &= -k x \\
	\\
	\text{Conservation of Energy (without transfer)} \qquad \Delta K +  \Delta U +  \Delta E_{th} + &= 0 \\
	\text{Conservation of Energy (with transfer)} \qquad \Delta K +  \Delta U +  \Delta E_{th} + &= w + q \\
	\text{Kinetic Energy:} \qquad K &= \frac{1}{2} mv^2 \\
	\text{Gravitational Potential Energy:} \qquad U_g &= mgy \\
	\text{Spring Potential Energy:} \qquad U_x &= \frac{1}{2} kx^2 \\
	\text{Work:} \qquad W &= F d (cos \theta) \\
	\text{Power:} \qquad P &= \frac{\Delta E}{\Delta t} \\
	\text{Mechanical Power:} \qquad P &= \frac{W}{\Delta t} = F \nu \\
	\text{Energy Efficiency:} \qquad e &= \frac{E_{out}}{E_{in}} \\
\end{align*}

\newpage


\subsection{Thermal Properties}

\begin{align*}
T &= \frac{2}{3} \frac{K_{avg}}{k_b} \\
\text{Thermal expansion (volume):} \qquad \Delta V &= \beta V, \Delta T \\
\text{Thermal expansion (linear):} \qquad \Delta L &= \alpha L, \Delta T \\
\text{Gas Pressure} \qquad p &= \frac{2}{3} \frac{N}{V} K_{avg} \\
\text{Ideal Gas Law (multiple forms)} \qquad p V &= N K_b T &= n N_A k_b T &= n R T  \\
\text{Mass of a Substance} \qquad  m &= \rho V \\
\text{Work Done by a Gas} \qquad W_{gas} &= p \Delta V \\
\text{Energy Conservation for Interacting Systems} \qquad Q_{net} &= Q_1 + Q_2 + \ldots = 0 \\
\\
\text{Heat Equations for Solids and Liquids} \qquad Q &= m C \Delta T \\
\\
\text{Heat Equations for Gasses}  \\
\text{Constant Volume} \qquad Q &= n C_v \Delta T \\
\text{Constant Pressure} \qquad Q &= m C _p\Delta T \\
\\
\text{Heat Required for a Phase Change}  \\
\text{Fusion (melting.freezing)} \qquad Q &= \pm M L_f \\
\text{Vapourization (boiling/conmdensing)} \qquad Q &= \pm M L_v \\
\\
\text{Rate of Conduction Across a Temperature Gradient} \qquad Q &= \frac{k A}{L} \Delta T \\
\\
\text{Rate of Radiative Heat Transfer} \qquad \frac{Q}{\Delta T} &= e \ \sigma A T^4 \\
\\
\end{align*} 

\newpage


\section{Fluids}

\begin{align*}
	\text{Fluid Pressure} \qquad p &= \frac{F}{A} \\
	\text{Hydrostatic Pressure} \qquad p &= p_0 + \rho g d \\
	\text{Gauge Pressure} \qquad p_g &= p - p_{atm} \\
	\text{Pressure Gradiaent in a Viscous Fluid} \qquad \Delta p &= 8 \pi \nu \frac{L V_{avg}}{A} \\
	\text{Buoyancy Force} \qquad F_b &= \rho_f V_f g \\
\end{align*}
	
\newpage


\section{Oscillations}

\begin{align*}
	\text{Frequency-period relationship} \qquad f &= \frac{1}{T} \\
	\text{Frequency of mass-spring oscillator} \qquad f &= \frac{1}{2 \pi} \sqrt{\frac{k}{m}} \\
	\text{Frequency of pendulum oscillator (small angle of displacement)} \qquad f &= \frac{1}{2 \pi} \sqrt{\frac{g}{L}} \\
\end{align*}

\newpage


\section{Waves}

\begin{align*}
	\text{Wave speed of a stretched string} \qquad v_{string} &= \sqrt{\frac{T_s}{\mu}} \\
	\text{Wave speed in a gas} \qquad v_{sound} &= \sqrt{\frac{\gamma R T}{M}} \\
	\text{Wave speed} \qquad v &= f \lambda = \frac{\lambda}{T} \\
	\text{Sound intensity} \qquad \beta &= (10 dB) log_{10}\big(\frac{I}{I_0} \big) I = \frac{P_{source}}{4 \pi r^2} \\
\end{align*}

\newpage


\section{Doppler Effect}

\begin{align*}
	\text{Observed frequency (source approaching at $v_s$)} \qquad f_{+} &= \frac{f_0}{1 - v_a/v} \\
	\text{Observed frequency (source receding at $v_s$)} \qquad f_{-} &= \frac{f_0}{1 + v_a/v} \\
	\text{Observed frequency (observer approaching source at $v_o$)} \qquad f_{+} &= \frac{1 + v_o}{v} f_0 \\
	\text{Observed frequency (observer receding from source at $v_0$)} \qquad f_{-} &= \frac{1 - v_o}{v} f_0 \\
\end{align*}

\newpage



\section{Optics}

\begin{align*}
	\text{Speed of light in a transparent mediun} \qquad v &= \frac{c}{n} \\
	\text{Snall's Law} \qquad n_1 sin \ \theta_1  &= n_2 sin \ \theta_2 \\
	\text{Critical angle (total internal reflection)} \qquad \theta_c &= sin^{-1} \Big( \frac{n_2}{n_1} \Big) \\
	\text{Optical magnification} \qquad m &= \frac{s^{'}}{s} = \frac{h^{'}}{h}\\
	\text{Len power} \qquad P &= \frac{1}{f} \\
	\text{Thin lens equation} \qquad \frac{1}{s} + \frac{1}{s^{'}} &= \frac{1}{f} \\
	\text{Light gathering ability} \qquad f - \text{number} &= \frac{f}{d} \\
	\text{Simple magnifier} \qquad M &= \frac{25 \  \text{cm}}{f} \\
\end{align*}

\newpage



\section{electric Fields and Forces}

\begin{align*}
	\text{Coulomb's law} \qquad F_{1 on 2} &= F_{2 on 1} = \frac{K \left| q_1 \right| \left| q_2 \right| }{r^2} \\
	\text{Electric field at point defined by charge $q$} \qquad E(x,y,z) &= \frac{F_{on\ q}(x,y,z)}{q} \\
	\text{xxx} \qquad p &= \frac{F}{A} \\
	\text{xxx} \qquad p &= \frac{F}{A} \\
	\text{xxx} \qquad p &= \frac{F}{A} \\
	\text{xxx} \qquad p &= \frac{F}{A} \\
	\text{xxx} \qquad p &= \frac{F}{A} \\
	\text{xxx} \qquad p &= \frac{F}{A} \\
	\text{xxx} \qquad p &= \frac{F}{A} \\
	\text{xxx} \qquad p &= \frac{F}{A} \\
	\text{xxx} \qquad p &= \frac{F}{A} \\
	\text{xxx} \qquad p &= \frac{F}{A} \\
	\text{xxx} \qquad p &= \frac{F}{A} \\
	\text{xxx} \qquad p &= \frac{F}{A} \\
	\text{xxx} \qquad p &= \frac{F}{A} \\
	\text{xxx} \qquad p &= \frac{F}{A} \\
\end{align*}

\newpage


\section{Quantum Numbers}

\begin{align*}
	\text{Bohr energy of an hydrogen atom} \qquad E_n &= \frac{13.6 \ eV}{n^2} \quad \text{where $n$ = 1, 2, 3, 4, ...} \\
	\text{Angular momentum of an electron's orbit} \qquad L &= \sqrt{\ell(\ell+1)}\hbar \quad \text{where $\ell$ = 0, 1, 2, 3, ..., n-1} \\
	\text{Magnetic quantum number} \qquad m &= -\ell, -\ell_{+1}, ....0, \ell_{-1}, \ell \\
	\text{Spin quantim number} \qquad m_s &= -\frac{1}{2} \ \text{or} \  +\frac{1}{2} \\
\end{align*}

%% https://tex.stackexchange.com/questions/231322/how-to-get-the-lowercase-calligraphic-symbols

\newpage


\section{Nuclear Physics}

\begin{align*}
	\text{Half life} \qquad N &= N_0 {\Big( \frac{1}{2} \Big)}^{\frac{t}{t_{1/2}}} \\
	\text{Exponential decay} \qquad N &= N_0 \ e^{-\frac{t}{\tau}} \\
	\text{Activity} \qquad R &= \frac{N}{\tau} = \frac{0.693 \ N}{t_{1/2}} \\
	\text{Binding energy} \qquad B &= ( Z m_H + N m_n - m_{atom} ) \times (931.49 \ M eV/u) \\
\end{align*}

\newpage


\section{Periodic Tables}


\end{document}
